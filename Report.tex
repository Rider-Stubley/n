\documentclass[12pt]{report}
\usepackage[utf8]{inputenc}

\renewcommand{\baselinestretch}{1}

\setcounter{tocdepth}{3}

\title{}
\author{Rebecca Green, Liam McMahon, Rider Stubley}

\usepackage{amsmath}
\usepackage{graphicx}
\usepackage{parskip}
\usepackage{float}
\usepackage[margin=25.4mm]{geometry}
\usepackage{gensymb}

\usepackage{times}
\usepackage{titlesec}
\usepackage[hidelinks]{hyperref}
\usepackage[font={small},rm]{caption}

\renewcommand{\abstractname}{Executive Summary}

\makeatletter
\def\@makechapterhead#1{%
	\vspace*{50\p@}%
	{\parindent \z@ \raggedright \normalfont
		\ifnum \c@secnumdepth >\m@ne
		\if@mainmatter
		%\huge\bfseries \@chapapp\space \thechapter
		\Huge\bfseries \thechapter.\space%
		%\par\nobreak
		%\vskip 20\p@
		\fi
		\fi
		\interlinepenalty\@M
		\Huge \bfseries #1\par\nobreak
		\vskip 18\p@
	}}
	\makeatother

\titleformat*{\section}{\bfseries\fontsize{14}{12}\sffamily}
\titleformat*{\subsection}{\itshape\bfseries\fontsize{12}{12}}

\begin{document}
	
	\pagenumbering{roman} % Makes the front matter of the report have Roman numerals. Plz leave
	\begin{titlepage}
		\begin{center}
			\vspace*{1cm}
			
			\textbf{METR3100 Sensors and Actuators}
			
			\vspace{0.5cm}
			Actuators Practical Aligned Assignment
			
			\vspace{1.5cm}
			\textbf{Rebecca Green, Liam McMahon, Rider Stubley} \\
			\vspace{0.5cm}
			(43213517, 43186022, 43205259)
			
			\vfill
			
			METR3100 \\
			University of Queensland \\
			Australia \\
			27/04/2015 \\
		\end{center}
	\end{titlepage}
	
	% Put executive summary here
	\begin{abstract}
		
	\end{abstract}
	
	\tableofcontents
	
	% Use \chapter as the first level, \section as the second, \subsection as the third
	\chapter{Introduction}
	\pagenumbering{arabic} % starts the numbering in Arabic numerals at the first chapter. Plz leave
	
	\section{Aims}
	
	\section{Scope} % ?
	
	\section{Contents of Report}
	
	\section{Contributions}
	
	\section{Background}
	
	\chapter{Equipment and Procedure}
	
	\section{Equipment}
	\subsection{Practical Equipment}
	The equipment used in the practical were:
	\begin{itemize}
		\item{AC motor}
		\item{Brake}
		\item{Cooling fan}
		\item{ABB drives}
		\item{PC with \textit{drivewindows}; and}
		\item{DSP7000}
	\end{itemize}
	\subsection{Safety Equipment}
	The safety equipment used in the practical were:
	\begin{itemize}
		\item{Enclosed shoes; and}
		\item{Hearing protection}
	\end{itemize}
	
	\section{Procedure}
	The procedure followed for this practical was:
	\begin{enumerate}
		\item{The ABB motor drives were turned on.}
		\item{\textit{Drivewindows} was started on the PC connected to the ABB motor drive.}
		\item{Remote control was taken over the motor.}
		\item{The control mode of the motor was switched to scalar.}
		\item{The frequency of the motor was set to 50 Hz.}
		\item{The cooling fan was started for the brake.}
		\item{The DSP7000 was started and set to open loop mode.}
		\item{The brake was turned on.}
		\item{The torque and speed displayed on the DSP7000 and the torque, speed, and motor current on \textit{drivewindows} and the information panel on the ABB drive were noted.}
		\item{The brake percentage was increased incrementally and the torque, speed, and motor current were noted for all brake percentages examined.}
	\end{enumerate}
	
	\chapter{Results}
	
	\chapter{Analysis and Discussion}
	
		\section{Experimental Data Analysis}
		%Breakdown of the data, errors etc.
		\section{Design Problem}
		%Basically Part D
		The scenario shown in Figure (insert number here) involves the ABB model AC motor pulling 		a mass up a given incline of $12.5\degree$. The motor shaft is attached to a flywheel of 
		radius $R=100mm$ and inertia $J=5kgm^2$
		%\begin{figure}[ht!] Basically I suck and can't insert the image here
		%	\centering
		%	\includegraphics[width=textwidth]{motorramp.png}
		%	\caption{Part D design scenario.}
		%\end{figure}
		The force opposing the torque of the motor is modelled primarily by the viscous friction model 		given by:
		\begin{align*}
			F_f=\gamma m \cos \theta \dot{x}
		\end{align*}
		Where $\gamma$ is a friction constant $=$ 0.1, $m$ is the mass of the block, $\theta$ is the 
		the angle of inclination of the block from horizontal and $\dot{x}$ is the speed the block is
		being pulled up the ramp. In this system, there is also a load component due to gravity.\\
		By taking the sum of the torques about the motor axis, we get the following relationship.
		\begin{gather*}
			\Sigma T_{axis} = J\dot{\omega} \\
			T_{load} - F_fR - F_{gravity}R=J\dot{\omega}\\
			T_{load}=J\dot{\omega}+F_fR+F_{gravity}R
		\end{gather*}
		It should be noted that for fixed speed operation, $\dot{\omega}=0$. Having expressions for 
		$F_f$ and $F_{gravity}$.
	\chapter{Conclusions}
	
	\chapter{Recommendations}
	
	\chapter{References}
	
	\chapter{Appendices}
	
	
\end{document}          
